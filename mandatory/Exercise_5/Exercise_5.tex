\documentclass[a4paper,10pt]{article}

% Packages.
\usepackage[T1]{fontenc}
%\usepackage{amsmath}
%\usepackage{amssymb}
\usepackage{listings}
%\usepackage[hidelinks]{hyperref} % Load last.

% Document.
\begin{document}
\title{Exercise 5}
\author{Marco Casari}
\date{\today}
\maketitle

\lstset{
  basicstyle=\ttfamily,
  breakatwhitespace=false,
  breaklines=true,
  columns=fullflexible,
  keepspaces=true,
%  language=[90]Fortran,
  numbers=left
}
\begin{lstlisting}
Declare parameters for: NAN value, seconds in a day, maximum number of consecutive NAN values.
Declare real variables for: input file name, sum of shortwave radiation, sum of infrared radiation, read value, two temporary variables for interpolation, slope, offset.
Declare integer variables for: input file IO state, read IO state, day, month, year, hour, minute, second, number of records, NAN counter, NAN index.
Insert input file name.
Trim filename.
Open input file for reading.
Check if there is an error opening file. If error, then.
  Write that there is an error.
If no error opening file, then.
  Set NAN counter to 0.
  Set first temporary variable to 0.
  Set number of records to 0.
  Set sum of shortwave radiation to 0.
  Set sum of infrared radiation to 0.
  Read header of input file.
  Loop.
    Read line.
    If no line left to read, then.
      Exit loop.
    Else if error reading file, then.
      Write that there is an error reading file.
      Exit loop.
    Else if NAN counter is greater than maximum number of consecutive NAN values, then.
      Write that maximum number of consecutive NAN values is exceeded.
      Exit loop.
    If read value is not NAN value, then.
      If NAN counter is greater than 0, then
        Set second temporary variable to read value.
        Evaluate and set slope.
        Evaluate and set offset.
        Loop NAN index from 1 to NAN counter stepping by 1.
          Set first temporary variable to interpolated value.
          If first temporary variable is non negative, then.
            Add first temporary variable to sum of shortwave radiation.
          Else.
            Add first temporary variable to sum of infrared radiation.
        Add NAN counter to number of records.
      Set first temporary variable to read value.
      Add 1 to number of records.
      If first temporary variable is non negative, then.
        Add first temporary variable to sum of shortwave radiation.
      Else.
        Add first temporary variable to sum of infrared radiation.
    Else.
      Add 1 to NAN counter.
  Evaluate and write daily energy of shortwave radiation.
  Evaluate and write daily energy of infrared radiation.
Close input file.
Exit program.
\end{lstlisting}
\end{document}
